%% 
%% Copyright 2007-2019 Elsevier Ltd
%% 
%% This file is part of the 'Elsarticle Bundle'.
%% ---------------------------------------------
%% 
%% It may be distributed under the conditions of the LaTeX Project Public
%% License, either version 1.2 of this license or (at your option) any
%% later version.  The latest version of this license is in
%%    http://www.latex-project.org/lppl.txt
%% and version 1.2 or later is part of all distributions of LaTeX
%% version 1999/12/01 or later.
%% 
%% The list of all files belonging to the 'Elsarticle Bundle' is
%% given in the file `manifest.txt'.
%% 
%% Template article for Elsevier's document class `elsarticle'
%% with harvard style bibliographic references

%\documentclass[preprint,12pt,authoryear]{elsarticle}

%% Use the option review to obtain double line spacing
\documentclass[authoryear,preprint,review,12pt]{elsarticle}

%% Use the options 1p,twocolumn; 3p; 3p,twocolumn; 5p; or 5p,twocolumn
%% for a journal layout:
%% \documentclass[final,1p,times,authoryear]{elsarticle}
%% \documentclass[final,1p,times,twocolumn,authoryear]{elsarticle}
%% \documentclass[final,3p,times,authoryear]{elsarticle}
%% \documentclass[final,3p,times,twocolumn,authoryear]{elsarticle}
%% \documentclass[final,5p,times,authoryear]{elsarticle}
%% \documentclass[final,5p,times,twocolumn,authoryear]{elsarticle}

%% For including figures, graphicx.sty has been loaded in
%% elsarticle.cls. If you prefer to use the old commands
%% please give \usepackage{epsfig}

%% The amssymb package provides various useful mathematical symbols
\usepackage{amssymb}
%% The amsthm package provides extended theorem environments
%% \usepackage{amsthm}

\usepackage{siunitx}  % for degree celsius
\usepackage{multirow} % for the table
\usepackage{booktabs} % fot toprule, midrule
\usepackage{longtable}

\usepackage{color,soul} % highlight 

% for track changes in revision version, comment this one if want the final version
%\usepackage{changes}
%\usepackage[final]{changes}

\usepackage{hyperref} %for breaking long url

%% The lineno packages adds line numbers. Start line numbering with
%% \begin{linenumbers}, end it with \end{linenumbers}. Or switch it on
%% for the whole article with \linenumbers.
%% \usepackage{lineno}

\usepackage{lineno}
\linenumbers

\journal{Remote Sensing of Environment}

\begin{document}

\begin{frontmatter}

%% Title, authors and addresses

%% use the tnoteref command within \title for footnotes;
%% use the tnotetext command for theassociated footnote;
%% use the fnref command within \author or \address for footnotes;
%% use the fntext command for theassociated footnote;
%% use the corref command within \author for corresponding author footnotes;
%% use the cortext command for theassociated footnote;
%% use the ead command for the email address,
%% and the form \ead[url] for the home page:
%% \title{Title\tnoteref{label1}}
%% \tnotetext[label1]{}
%% \author{Name\corref{cor1}\fnref{label2}}
%% \ead{email address}
%% \ead[url]{home page}
%% \fntext[label2]{}
%% \cortext[cor1]{}
%% \address{Address\fnref{label3}}
%% \fntext[label3]{}

%\ead{huanglingcao@link.cuhk.edu.hk}

\title{Automatically quantifying evolution of retrogressive thaw slumps in Beiluhe (Tibetan Plateau) using multi-temporal CubeSat images}

%\title{Mapping Retrogressive Thaw Slumps in the Beiluhe Region (Tibetan Plateau) from CubeSat Images using Deep Learning}

%% use optional labels to link authors explicitly to addresses:
%% \author[label1,label2]{}
%% \address[label1]{}
%% \address[label2]{}

\author[a]{Lingcao Huang}
\author[a]{Lin Liu}
\author[b]{more}
%\author[b]{Jing Luo}
%\author[b]{Zhanju Lin}
%\author[b]{Fujun Niu}



\address[a]{Earth System Science Programme, Faculty of Science, The Chinese University of Hong Kong, Hong Kong SAR, China.}
\address[b]{Add a few more co-authors?}
%\address[b]{Northwest Institute of Eco-Environment and Resources, Chinese Academy of Sciences, Lanzhou, China.}

\begin{abstract}

Retrogressive thaw slumps (RTSs) are among the most dynamic landforms resulting from the thawing of ice-rich permafrost. 
However, their distribution and evolution are poorly quantified due to the challenges in accurately mapping them from remote sensing images. 
Building on the previous work on mapping RTSs using deep learning, we apply the same method to multi-temporal images, aiming to quantify RTS development in different years. 
%Firstly, we selected 36 active RTSs in Beiluhe on the Tibetan Plateau and manually delineated their boundaries on Planet CubeSat images taken in July 2017, 2018, and 2019. 
%Secondly, we derived training data from these boundaries and Planet images, then trained the DeepLabv3+ model. 
%Thirdly, we predicted RTSs in 2017, 2018, and 2019 Planet images using the well-trained model. Lastly, we conducted polygon-based change detection (2017 vs. 2018 and 2018 vs. 2019) on the mapped RTSs in different years, then obtained change polygons corresponding to the expanding areas toward upslope. 
%The results show that from 2017 to 2018, there are 35 change polygons, and their areas range from 108 $m^2$ to 2273 $m^2$ with a mean of 1284 $m^2$; 
%from 2018 to 2019, 31 change polygons are obtained, and their minimum, maximum, and average areas are 142 $m^2$, 9360 $m^2$, and 1913 $m^2$, respectively. 
%Visual inspection shows that most change polygons can accurately outline the expanding areas, yet a few of them are false positives, and a few expanding areas are missed. 
One advantage of using multi-temporal images to map RTSs then quantify their development is that this strategy can significantly reduce incorrect mapping results because a specific incorrect one would unlikely occur at the same location across multi-temporal images. 
However, outlining the expanding areas requires very high accuracy of delineating RTSs in different images. 
This study demonstrates that the deep-learning-based mapping method can potentially apply to large areas for mapping RTSs and quantifying their development if high-resolution remote sensing images are available.  


\end{abstract}

\begin{keyword}
%% keywords here, in the form: keyword \sep keyword

%% PACS codes here, in the form: \PACS code \sep code

%% MSC codes here, in the form: \MSC code \sep code
%% or \MSC[2008] code \sep code (2000 is the default)
CubeSat  \sep Deep Learning \sep Permafrost \sep Retrogressive Thaw Slumps \sep   Remote Sensing.

\end{keyword}

\end{frontmatter}

%% \linenumbers
%\tableofcontents

%% main text
\section{Introduction}
\label{sec_intro}

% some background
Permafrost is the ground that remains 0$^\circ$C for at least two consecutive years and is warming at the global scale \citep{biskaborn2019permafrost}. 
Permafrost warming can cause permafrost degradation including active layer thickening, development of thermokarst landforms, shrinking of permafrost extent \citep{czudek_thermokarst_1970,jorgenson_response_2005,osterkamp2007Characteristics,aakerman2008thawing,zhao2010Thermal}. 
Furthermore, it can threaten the safety of infrastructure, release greenhouse gases, and alter the local ecosystem \citep{tong_effect_1996,yang2010permafrost,bowden2010climate,grosse_vulnerability_2011,vonk2015reviews,schuur_climate_2015,olefeldt_circumpolar_2016,schuster2018permafrost,hjort2018degrading}.


% Introduction on the retrogressive thaw slumps
A retrogressive thaw slump (RTS) is one of dynamic landforms resulting from the thawing of ice-rich permafrost \citep{czudek_thermokarst_1970, jorgenson_thermokarst_2013,farquharson2016spatial,jones2019rapid}. 
Usually, RTSs are triggered by lateral stream erosion or active layer detachments \citep{french2017periglacial} and can be active for decades \citep{burn1989geomorphology, lacelle2010climatic, swanson2018growth}. 
For instance, a detachment slide occurs due to the melting of ground ice or intensive precipitation and removes soil above permafrost, then exposes it to rapid thawing and initiates an RTS. 
The main component of an RTS are the headwall, slump floor, and slump lobe \citep{lantuit_fifty_2008}. 
At the headwall, the exposed permafrost can continue thawing in each summer, which would further expands the thawed area towards upslope \citep{french2017periglacial}. 
Moreover, RTSs can significantly alter the local ecosystem such as mass wasting of soil as well as vegetation \citep{gooseff2009effects} and increases of mercury concentrations in the downstream aquatic environment \citep{pierre2018unprecedented}.


% literature on the occurrence and development of RTSs.
The occurrence and dynamics of RTSs have been reported in many local permafrost areas, but their development in most areas is still unknown. 
The RTS occurrence has been reported in northern and central Alaska (e.g., \citealp{swanson2018growth,balser2014timing}), northern Canada (e.g., \citealp{burn1990canadian, cassidy2017impacts, armstrong2018thaw,lewkowicz2019extremes}) , Siberia (e.g., \citealp{leibman2003dynamics, zwieback2018sub}) and the Tibetan Plateau (e.g., \citealp{niu2005engineering, niu2016thaw}). 
As reported in some local or regional regions, their number and affected areas also increased dramatically in recent decades \citep{luo2019recent, lewkowicz2019extremes}.
Typical, their retreat rates are 6--8 m yr$^{-1}$ \citep{jorgenson_thermokarst_2013}, but vary in different regions such as 16 m yr$^{-1}$ reported in Yukon Territory \citep{burn1989geomorphology} and 38 m yr$^{-1}$  in Alaska \citep{swanson2018growth}.
However, the occurrence and retreat rates of RTSs in most permafrost areas are still unknown, especially on the Tibetan Plateau. 
%Possible reasons include (1) RTSs scatter in remote and inaccessible regions, which makes field measurements challenging and (2) their small size as well as similarity to the surroundings presents difficulties for the automatic algorithms utilizing remote sensing images.  

% why we need an automatically method. 
An automatic method is essential for quantifying RTS development throughout permafrost areas. 
Permafrost underlays approximately one-quarter of the land surface in the Northern Hemisphere \citep{zhang1999statistics} and most areas are remote and inaccessible. 
Therefore, it is impossible to conduct field measurement for RTS development. 
The advance of remote sensing satellites, especially, the CubeSat constellation provides multi-temporal and high-resolution images covering the global land surface. 
Manual delineation on these images can be used to quantify RTSs development but is impractical due to large areas and the huge volume of images as well as its characteristics of time-consuming and labor-intensive. 
Therefore, automatic methods are necessary for addressing the unknown questions related to RTS development throughout permafrost areas. 


% Motivation
Combining the deep learning algorithms and multi-temporal CubeSat images can automatically quantify the RTSs development.
A previous work has utilized deep learning algorithms to automatically delineate RTSs and output accurate mapping results \citep{huang2020using}. 
We can apply the same mapping method to multi-temporal images then delineate the RTSs in multi-temporal images. 
By comparing these multi-temporal RTS boundaries, we can quantify the development of these RTSs and obtain their retreat rates. 

%Evolution of retrogressive thaw slumps is poor quantified in most permafrost areas. 
%NO study on mapping RTS development automatically. 

% Objectives, outline of this paper and contributions 
The objectives of this study are (1) to extend the automatic mapping method proposed by \cite{huang2020using} to multi-temporal images and (2) to develop a method to automatically quantify the RTS development based on multi-temporal mapping results and obtain their retreat rates. 
We train a single DeepLabv3+ model using multi-temporal images then use this model to predict RTS boundaries in different years.
Then we propose a polygon-based change detection to remove incorrect mapping results and obtain change polygons covering expanding areas.  
We calculate the medial axis of each change polygon, then based on the medial axis to obtain the retreat rates.  
%We also run the k-fold (k=3,5,10) cross-validation to evaluate the robustness of mapping method when applying to multi-temporal images and conduct the statistics of RTSs development in Beiluhe. 
To the best of our knowledge, this would be the first study that provides an automatic method for quantifying RTS development on remote sensing images and very first study reporting retreat rates of RTSs on the Tibetan Plateau. 

%Contribution of this work: (1) provide an automatic method for calculating the retreat distance (no matter the RTS boundaries are from manual delineation or automatic mapping). 
%(2) What’s the IOU value of mapping polygons are required for automatic change detection? For polygon-based change detection, we also can calculate F1 score, but it rely on the accuracies of mapping results. \\\\



\section{Study area and data}
\label{sec_studyarea_data}

% Study area
Our study area is the Beiluhe region ($92.50^\circ$E to $93.51^\circ$E, $34.69^\circ$N to $35.18^\circ$N) on the Tibetan Plateau as shown in Fig. \ref{fig_multi_rts_image_studyarea}a.
The elevation is between 4418 and 5320 m; mean annual air temperature is \SI{-3.8}{\celsius}; annual precipitation is around 300 mm  \citep{luo_thermokarst_2015}.
It is underlain by ice-rich permafrost with thickness ranges from 20 to 80 m: 70\% and 20\% of this region has ice content greater than 30\% and 50\%, respectively \citep{zhou_geocryology_2000, luo_thermokarst_2015}; 
the type of ground ice is mainly massive ice \citep{guodong1983mechanism}. 
The active layer thickness and mean annual ground temperature ($\sim$15 m) ranges from 1.5 to 2.0 m and \SI{-2.0} to \SI{-0.5}{\celsius}, respectively \citep{zhou_geocryology_2000, wu2010changes, luo_thermokarst_2015,  wu2015changes}. 
As reported by a few studies, more than 200 RTSs developed in this region and their number and affected areas increased dramatically in the past decade \citep{huang2020using,luo2019recent}.  
For example, an RTS developed toward upslope from 2017 to 2019, and the development is captured by Planet CubeSat as shown in Fig. \ref{fig_multi_rts_image_studyarea} b to d. 
%We also choose a region (i.e., Test area) next to Beiluhe as shown in Fig. \ref{fig_multi_rts_image_studyarea}a for testing our automatic method. 
%The test area has 50 RTSs \citep{huang2020using} but does not have more detailed ground information. 

% Area for testing, should mention it here? Mentioned as above. 
%Test area: 50 RTS \citep{huang2020using}.

%TODO: plot zoom in of the headwall. 

%[ht]
\begin{figure} 
	\centering
	\includegraphics[width=14cm]{figs/rts_multi_images_study_area_v2_trim.jpg}
	\caption{The study area and development of a retrogressive thaw slump (RTS). The red and black rectangle in (a) shows the extent of our study area and test area on the Tibetan Plateau. (b) to (d) are Planet images (2017, 2018, and 2019) of an RTS whose central location is 92.912$^\circ$ E, 34.848$^\circ$ N. The white arrows point the position of the headwall. In (b), the yellow arrow indicates the upslope direction.}
	\label{fig_multi_rts_image_studyarea}
\end{figure}


%Data
The data include images from Planet CubeSat and RTS boundaries in July 2017, 2018, and 2019.
%address why only choose the images starting from 2017? 
We only use images after 2017 because images from Planet CubeSat constellation (Planetscope) are available after that time in our study area.  
In July, vegetation reaches its maximum coverage, and active RTSs are on the rapid stage of thawing processes. 
Therefore, identifying RTSs would be relatively easy during this period.
These images have four bands (blue, green, red, and near-infrared) with the spatial resolution of 3 m. 
Each image covers an approximate area of 10 km by 30 km.
Table \ref{table_image_list} lists the count and acquisition date of these images from 2017 to 2019. 
Because of the variation of cloud cover and image coverage, the image count in different years are different.
The RTS boundaries are derived from manual delineation on Planet images using the same method proposed by \cite{huang2020using}, parts of which will be used for training our automatic method. 
We consider all the RTS boundaries as ground truths in the validation step, and 70\% of them are cross-checked in the field in 2014 and 2018. %on other satellite images? 
%The steps of downloading as well as pre-processing of Planet images and collecting ground truths are presented by \cite{huang2020using}.


\begin{table}[ht]
\footnotesize
\caption{Count and acquisition date of images from the Planetscope satellites.}
\label{table_image_list}
\centering
%% \tablesize{} %% You can specify the fontsize here, e.g. \tablesize{\footnotesize}. If commented out \small will be used.
%\begin{tabular}{cc m{2.2cm}  m{2.2cm}  m{2.2cm} ccc}
\begin{tabular}{c  c  m{2.5cm}   c m{3.0cm} }
\toprule
 \multirow{2}{*}{ \textbf{Year/Month}}  & \multicolumn{2}{c}{ \textbf{Beiluhe}} &  \multicolumn{2}{c}{ \textbf{Test Area}}\\
 & Count & Acquisition date & Count & Acquisition date \\
\midrule
 2017 July & 71 & 17, 19, and 20 & 36 & 19 and 20 \\
 2018 July & 52 & 20, 24, and 25 & 33 & 15 and 16 \\
 2019 July & 46 & 25, 27, and 30 & 53 &  8, 24, 25, 27, and 30\\

\bottomrule
\end{tabular}

%\raggedright \#: the continuous experiment number
\end{table}


\section{Methods}
\label{sec_meth}

%TODO: add the whole flowchart.
% The major steps includes pre-processing of multi-temporal Planet images, training, prediction,  polygon-based change detection, and post-processing. 
%training on multi-temporal images, prediction on multi-temporal images, then polygon-based change detection: (1) removing false positives based on multi-temporal mapping results, (2) polygons based change detection based and got the expanding areas. 


\subsection{Pre-processing of multi-temporal images}
\label{sec_preprocessing}

We collected and pre-processed Planet CubeSat images before applying deep-learning-based mapping methods. 
We downloaded Planet images on Planet website (\url{www.planet.com}) via its Education and Research Program \citep{team2017planet}. 
For the images of each year, we followed the same steps presented by \cite{huang2020using} to pre-process them and obtained a mosaic covering the study area. 
%The steps of downloading, band extracting, stretching, and sharpening of these images in different years are presented by \cite{huang2020using}. 
%Co-registration
To increase the relative positional accuracy among multi-temporal images, we apply co-registration to the three mosaic images from 2017 to 2019. 
Firstly, we chose the mosaic image in \cite{huang2020using} as the reference image because it has been well pre-processed. Secondly, we used the scale-invariant feature transform algorithm \citep{lowe2004distinctive} with a GPU implement (\url{https://github.com/pitzer/SiftGPU}) to find tie-points between other mosaic images and the reference image \citep{huang2016a}. 
Lastly, we warped these images by using the Geospatial Data Abstraction Library (GDAL) with a first-order polynomial model.
%TODO: add co-registration accuracy in the supplementary files, also mentioned original 10 m accuracy 

% test histogram normalization in pre-processing steps (as noted, the histogram normalization did not improve the quanlity of Planet images, so we may just skip further steps)


\subsection{Delineating RTSs in different years}
\label{sec_delineating}

We used a deep-learning-based algorithm \citep{huang2020using} to automatically delineate RTSs on multi-temporal images.
Compared with the method in \cite{huang2020using},  the innovation of this work includes (1) training without using negative training polygons and (2) improving RTS mapping results by utilizing RTS dynamic information in multi-temporal images. 
The details will be presented as the following.

% not include negative training polygons, has bad results, but can be removed by the multi-temporal information. 

\subsubsection{Preparing training data} 
\label{sec_prepare_training}

%\subsubsection{Training a single model} 
%\label{sec_training_model}

% Preparing training data
The training data were derived from training polygons and Planet images in 2017, 2018, and 2019. 
We considered all the ground truths as positive training polygons but did not include any negative training polygons. 
Negative training polygons are necessary for distinguish land cover that are similar to RTSs but need ground information or some initial tests to generate them. 
However, generating negative training polygons will be impractical for mapping RTSs in a large area. 
Without negative training polygons, the mapping results may contain a lot of false positives, but we can utilize the RTS dynamic information in multi-temporal images to remove them (see section \ref{sec_improving_using_rts_dynamic}).
For the training polygons and Planet images in each year, we extracted sub-images with a buffer area of 300 m. 
Then we subdivided all sub-images into patches by setting a patch size as 480 pixels with an overlap of 160 pixels. 
Details of the extracting and subdividing operation are presented by \cite{huang2018automatic}.
To improve the volume of training data and generalization of the trained model, we applied data augmentation (flipping, blurring, cropping, and scaling) to the patches. 
Lastly, we merged all the patches from different years to a single training dataset. 


\subsubsection{Training a single model and predicting RTSs in different years}
\label{sec_train_prediction}

% training
We trained a deep-learning model, i.e., DeepLabv3+ \citep{chen2018encoder-decoder}, by using the training data derived in section \ref{sec_prepare_training}. 
\cite{huang2020using} have proofed that a well-trained DeepLabv3+ can delineate RTSs and obtain their boundaries that are very close to manual delineation. 
In this study, instead of training different models for different years, we only trained a single model because DeepLabv3+ has a high capability to represent RTS features in multi-temporal images, more importantly, a generic model can save a lot of computing resources, especially when expanding the time domain.
We used the same hyper-parameter presented in \cite{huang2020using} to train this model. 
%To evaluate the robustness of the model, we utilized k-fold cross-validation (k=3, 5, and 10)  as commonly used in machine learning. 
 %TODO: How about the test Area? use which trained model?
%k-fold cross-validation. 

% prediction
We used the trained model to predict RTSs on in 2017, 2018, and 2019 mosaic images. 
The steps of predicting RTSs in one year include (1) dividing it to numerous small patches, 
(2) predicting RTS or non-RTS pixels, 
(3) mosaicked the binary patches using GDAL and obtained a binary mosaic, 
(4) polygonizing the binary mosaic, and 
(5) removing polygons smaller than 0.3 ha. 
The size and overlap of each patches for prediction is the same to the one for preparing training data (section \ref{sec_prepare_training}). 
To improve efficiency, we adopted a parallel strategy to predict RTSs in different years.


\subsubsection{Improving RTS mapping results using the RTS dynamic feature}
\label{sec_improving_using_rts_dynamic}

Many mapped polygons were erroneously identified as RTS polygons can be removed by utilizing the RTS dynamic feature. 
As shown in Fig. \ref{fig_rts_expanding}, an active RTS can expand toward upslope annually, even for a non-active RTS, its extent cannot shrink. 
Therefore, we can use this dynamic feature to remove non-RTS mapped polygons. 
Firstly, we obtained a union (Fig. \ref{fig_rts_expanding}) of mapped polygons at the same location.
The steps of obtaining the union include (1) choosing a mapped polygon as initial union, (2) scanning through all mapped polygons until find a mapped polygon that is still available (not merged to other unions) and overlaps the union, (3) merging the found mapped polygon to union and updating the union, and (4) repeating steps (2) and (3) until cannot find a new mapped polygon can be merged to the union. 
Secondly, we calculated the intersection over union (IOU) for mapped polygons in each year (termed as ``time-IOU") as follows, 
\begin{equation}
IOU(A,B)_{year}=area(A \cap B)/area(A \cup B)
\label{equ_time_iou}
\end{equation}
where \emph{A} is a mapped polygon in the specific year and \emph{B} is the union. 
Due to the dynamic of RTSs, the time-IOU of an active RTS will monotonically increase over time. 
If the RTS is already stabilized, its time-IOU will remain the same over time. 
In consideration of the delineation error (e.g., the RTS boundaries in different years at non-headwall parts are not exactly overlap each other as shown in Figs. \ref{fig_rts_expanding} and \ref{fig_rts_change_det}), a threshold of $-0.1$ was set to determine the time-IOU of an RTS is monotonically increasing, that is, if all the differences ($IOU_{next\_year}-IOU_{year}$) are greater than $-0.1$, we consider the time-IOU is monotonically increasing. 
Thirdly, in the Beiluhe region, almost all the RTSs were initiated in or before 2016 \citep{luo2019recent}.
Therefore, we can assume that at a specific location where an RTS emerges, it presents three mapped polygons and each in 2017, 2018, and 2019. 
Lastly, we removed mapped polygons in a location that do not satisfy the three-polygon assumption or the time-IOU is not monotonically increasing.
 
%[ht]
\begin{figure} 
	\centering
	\includegraphics[width=14cm]{figs/rts_expanding_example_trim.jpg}
	\caption{RTS polygons (a) representing their boundaries from 2017 to 2019 and the union (b) of these polygons. The background is 2019 Planet images, and the RTS is the same to the one shown in Fig. \ref{fig_multi_rts_image_studyarea}}
	\label{fig_rts_expanding}
\end{figure}


\subsubsection{Validating mapped polygons}
\label{sec_validate_mapped_polygons}

To validate mapped polygons, we calculate F1 scores \citep{huang2020using} for mapped polygons in each year separately. 
For each year, we calculate IOU values between mapped polygons and the corresponding ground truths. 
We set a threshold of 0.5 to decide a mapped polygon is a false positive (FP) or true positive (TP).
After obtaining the count of true positives, false positives, and false negatives (FN), we can calculate the precision and recall, then calculate the F1 score using the following euqation. 
\begin{equation}
F1=2 \times Precision \times Recall / (Precision + Recall)
\label{equ_f1score}
\end{equation}
More details of the F1 score and its implication can be found in \cite{huang2020using}.
We repeated the calculation of F1 scores for different years and obtained the corresponding values. 

To assess the performance of the delineating method in this paper, we calculated the average precision (AP) values for mapping results in different years, then averaged them. 
AP values are based on precision-recall curves which is an effective metric for evaluating the algorithm performance \citep{huang2020using}.
A higher average of AP values indicates the better performance of the method. 

\subsection{Detecting changes of RTS boundaries}
\label{sec_detect_rts_changes}

This study will provide a method to automatically detect changes of RTS boundaries and calculate their annual retreat distance.
The RTS boundaries are in vector format and can be derived from manual delineation or automatic mapping. 
The main steps include polygon-based change detection, removing false changes, and calculating retreat distance as follows, 

\subsubsection{Polygon-based change detection}
\label{sec_polygon_change_det}

%a figure showing a multiPolygon and its narrow parts. 
%[ht]
\begin{figure} 
	\centering
	\includegraphics[width=14cm]{figs/rts_polygon_change_det_trim.jpg}
	\caption{An example of the polygon-based change detection. (a) shows the boundaries of an RTS (Fig. \ref{fig_rts_expanding}a) in 2017 and 2018. $P_{2018}-P_{2017}$ results in an expanding multiPolygon (b) and the corresponding non-narrow parts (c). $P_{2017}-P_{2018}$ results in a shrinking multiPolygon (d).}
	\label{fig_rts_change_det}
\end{figure}

To obtain the annual RTS changes, we conducted change detection for RTS boundaries between this year and the previous year, that is, comparing 2017 with 2018 and 2018 with 2019. 
 As shown in Equation \ref{equ_polygon_diff}, for a specific location has mapped polygons in different years, a polygon in year \emph{t} minus the one in year \emph{t-1} can result in an expanding polygon or multiPolygon. 
We used the ``difference'' function in Shapely (\url{https://pypi.org/project/Shapely}) to implement the subtraction in Equation \ref{equ_polygon_diff}.
\begin{equation}
E= P_{t} - P_{t-1}
\label{equ_polygon_diff}
\end{equation}
where \emph{P} is an RTS polygon, \emph{t} represents year, and \emph{E} is the expanding polygon or multiPolygon. 
While $P_{t} - P_{t+1}$ can result in a shirking polygon, but it does not correspond to the dynamic feature of an RTS; instead, it is a result of delineation errors. 
Fig. \ref{fig_rts_change_det} shows an example of the subtraction operation between the RTS boundaries in 2017 and 2018. 
%removing the narrow parts. 
With the consideration of delineation errors and the resolution (3 m) of Planet images, we removed the narrow parts of expanding multiPolygons using buffer operation as follows, 
\begin{equation}
P_{non\text{-}narrow}= P.\text{buffer}(-r).\text{buffer}(r*f) \cap P
\label{equ_polygon_buffer}
\end{equation}
where \emph{P} is a polygon or multiPolygon, \emph{r} is the value of buffer operation,  \emph{f} is a factor between 1 to 2 for avoiding uncertainties of buffer operation. 
%TODO: check or update the r and f value in the codes
In this study, we set \emph{r} as 3 and \emph{f} as 1.6 after initial experiments and used ``buffer'' function in Shapely.
To make it practical for calculating polygon attributes (e.g., area and parameter) and removing false changes (section \ref{sec_removing_false_change}), we converted all the multiPolygons to Polygons (termed as change polygons).



%a multi-polygon to polygons and calculate the attributes: area, circularity, 
%zeroIntersection, two or more intersections, holes, relative elevation, retreat distance. 

\subsubsection{Removing false change polygons}
\label{sec_removing_false_change}

%TODO: in this section, some threshold for removing operation need to confirm in codes or more experiments. 
%Polygon change detection. \\\\
This study utilized pre-set criteria to remove change polygons that are possibly incorrect. 
%TODO: check using ha or m^2
Firstly, we removed a change polygon if its size smaller than 36 $m^2$ or greater than 50000 $m^2$. 
%minimum_change_area = 36
%maximum_change_area = 50000
Because (1) we assumed that minimum changes on Planet images are $2\times2$ pixels, that is, 36 $m^2$ and (2) as reported by \cite{huang2020using}, 90\% RTSs in Beiluhe are smaller than 50000 $m^2$. 
%b_remove_polygons_with_holes = Yes
% bad mapping results, actually, but can not remove a few because previous steps have remove mapped polygons based on multi-temporal images. 
%TODO: need to discuss this in the discussion session.
%b_remove_zero_intersection = Yes
%b_remove_two_or_more_intersection = Yes
Secondly, a change polygon derived from erroneous mapped polygons will be removed.
We considered a mapped polygon as erroneous if it meets one of the following criteria: (1) with holes, (2) does not intersect any mapped polygons in other years, and (3) intersects two or more mapped polygons in other years.
Usually, these erroneous mapped polygons are derived from automatic mapping methods. 
%minimum_relative_elevation = 0
Thirdly, we removed change polygons if their elevation difference (Equation \ref{equ_elevation_diff}) is smaller than zero. 
\begin{equation}
Elevation_{diff}=Elevation_{ChangePolygon_{(P_{t}-P_{t-1})}} -  Elevation_{P_{t-1}}
\label{equ_elevation_diff}
\end{equation}
We used the 30 m DEM from Shuttle Radar Topography Mission (SRTM) \citep{farr2007shuttle} for calculating the elevation difference. 
For a (change) polygon, We chose the mean of elevation within it as its elevation. 
%minimum_retreat_distance = 6
Lastly, we set a threshold of 6 m for retreat distance (see section \ref{sec_cal_retreat_dis}) to remove change polygons with the consideration of Planet image resolution and delineation errors. 
%minimum_change_circularity = 0 (no effect )


\subsubsection{Calculating and validating retreat distance}
\label{sec_cal_retreat_dis}

% To calculate the retreat distance
We utilized medial circles, the inscribed circles contained in a change polygon and tangent to it in at least two points, to calculate the annual retreat distance of RTSs. 
A series of medial circles can represent the shape of a polygon, and their centers indicate the medial axis of the polygon \citep{zhu2014computing}, such as the circles and red lines in Fig. \ref{fig_retreat_dis}a. 
% as illustrated in Fig \ref{fig_retreat_dis},
Among these medial circles, the diameter of the maximum one is a good approximation (Fig. \ref{fig_retreat_dis}b) for the retreat distance.
However, due to the irregular and complicated shapes of change polygons, medial circles may underestimate the maximum retreat distance and cannot handle multiple retreat branches as well as ones with extreme elongated shape. 
To assess the underestimation and provide supplement information of retreat distance, we automatically draw lines crossing the center of the maximum medial circle to intersect the change polygons and calculate the distance based on these intersected line segments, as illustrated in Fig .\ref{fig_retreat_dis}c. 
The direction of these lines can be determined by (1) the direction of the maximum elevation difference and (2) from the centroid of the corresponding old polygon to the center of the circle. 
The retreat distance based on the line segments tend to be overestimated but can provide additional information because their accuracies are highly sensitive to elevation accuracy and shape of the RTS boundary in the previous year (e.g., the old polygon in Fig. \ref{fig_retreat_dis}). 
The variation of these three distances can indicate the consistency of the three measurement mentioned above . %A small value of difference indicate high confidence, and vice versa. 
We calculated the standard deviation of the three distance for each RTS and used it as an indicator for confidence of automatic calculation of retreat distance. 
Usually, the higher variation are due to irregular shapes of change polygons. 
For the case when an RTS contain several change polygons, we chose the maximum value as the RTS retreat distance.
%We calculate the variation based on the following equation: 
%\begin{equation}
%PlaceholderFOR=dis_{confidence}
%\label{equ_dis_confid}
%\end{equation}

%a figure showing a multiPolygon and its narrow parts. 
%[ht]
\begin{figure} 
	\centering
	\includegraphics[width=14cm]{figs/retreat_distance_trim.jpg}
	\caption{Calculating the retreat distance from a change polygon based on its medial axis and circles. (a) shows the medial axis (red lines) and some medial circles as well as circle centers (greed points). (b) shows the maximum medial circle whose diameter can be considered as the retread distance. Other two approaches: along the direction of maximum elevation difference and from the centroid (the red point in (c)) of the old polygon to the center of the maximum medial circle, are shown as yellow and red lines in (c), respectively. The background in (c) is the SRTM elevation, and the brighter color indicates higher elevation. (d) shows a manually drawn line pointing the retreat direction, and based on which to derive the retreat distance $L$. }
	\label{fig_retreat_dis}
\end{figure}

% We used the algorithm medial circle \citep{zhu2014computing}.

% TODO: Use the medial circle directly, for the one with lines along maximum elevation difference or the old polygon center, may put them into the supplementary files. 

%Step 1: got medial circles, sorted the circles by radius, resample the circles by  the data and find out what is the results. 

%We calculated the retreat distance for each change polygon based on its medial axis. 

% manually calculate the retreat distance by drawing a line indicating toward slope direction. 
% an RTS may have two or more branches, group RTS change polygons belongs to the same polygons. 
To validate the retreat distance, we manually draw line indicating expanding direction, and based these lines to calculate retreat distances as ground truth, as shown in Fig. \ref{fig_retreat_dis}d. 
For the case where an RTS has more than one expanding branches, we use chose the maximum one as its retreat distance. 



%\subsubsection{Validating RTS changes}
%\label{sec_validation_change}

\section{Results}
\label{sec_result}

%\subsection{The mapping performance when using multi-temporal images}
\subsection{Mapping results in multi-temporal images}
\label{sec_mapping_performance}

\subsubsection{Mapped polygons in 2017, 2018, and 2019}
\label{sec_mapped_polygons}

The mapped polygons accurately delineate most RTS boundaries on images taken in 2017, 2018, and 2019, with F1 scores of 0.673, 0.680, and 0.678, respectively.
Among the mapped polygons, around two-third of them are true positives (black, blue, and red polygons in Fig. \ref{fig_mapping_results}), and remains are false positives (yellow polygons in Fig. \ref{fig_mapping_results}). 
Table \ref{table_accuracy_f1} lists number of true positives, false positives, and false negatives, as well as precision, recall, and F1 scores in each year, with and without applying RTS-dynamic features.
The distribution of mapped polygons for the entire Beiluhe region is shown in Fig. \ref{fig_mapping_results}a, amplifications of three locations are shown in Fig. \ref{fig_mapping_results}b, c, and d. 
Compared with ground truths presented in Fig. S1 in the Supplementary Materials,  around 70 RTSs are missed in the mapped polygons such as the two RTSs marked by the yellow rectangle in Fig. \ref{table_accuracy_f1}b.
As shown in Fig. \ref{table_accuracy_f1}c and d, the true positives are close to the ground truths, that is, the automatically delineated ones are similar to the ones from manual delineation. 

%%%%%%%%%%%%%%%%%%%%%%%%%%%%%%%%%%%%%%%%%%%%% 
%%%% put the mapped polygons in Test Area to Supplementary Materials??? %%%%
%%%%  However,  the results for test area are not good (F1 score around 0.1), maybe it's ok to exclude the test area in this manuscript. The focus of this manuscript is the polygon-based detection, not mapping, the get the retreat distance automatically. Also, the performance of Deeplabv3+ has been proved in the previous RES paper.
%%%%%%%%%%%%%%%%%%%%%%%%%%%%%%%%%%%%%%%%%%%%%%



%%%%%%% Mapping results from exp7 %%%%%%%%%%%%%
\begin{figure} 
	\centering
	\includegraphics[width=14cm]{figs/multi_mapping_results_trim.jpg}
	\caption{Mapped polygons in different years after applying time-IOU. The black, blue, and red polygons are truth positives delineated on 2017, 2018, and 2019 images, respectively, while the yellow ones are false  positives. The background is the mosaic of 2019 Planet images. (a) shows the results in the entire Beiluhe region, and (b)--(d) are amplifications of three regions marked by rectangles in (a). }
	\label{fig_mapping_results}
\end{figure}

%%%%%%% Mapping results from exp7 %%%%%%%%%%%%%
\begin{table}[ht]
\footnotesize
  \centering
  \caption{Accuracy of mapped polygons in 2017, 2018, and 2019, respectively, with and without considering RTS dynamic features. TP, FP, and FN are number of true positives, false positives, and false negatives.}
    \begin{tabular}{c c c c c c c}
\toprule
    \textbf{Year} & \textbf{TP} & \textbf{FP} & \textbf{FN} & \textbf{Precision} & \textbf{Recall} & \textbf{F1 score} \\
\midrule
   2017 & 224   & 2493  & 42    & 0.082 & 0.842 & 0.150 \\
   2018 & 234   & 1863  & 32    & 0.112 & 0.880 & 0.198 \\
   2019 & 231   & 1579  & 36    & 0.128 & 0.865 & 0.222 \\
   $2017_{RTS{\text -}dynamic}$ & 192   & 113   & 74    & 0.630 & 0.722 & 0.673 \\
   $2018_{RTS{\text -}dynamic}$ & 194   & 111   & 72    & 0.636 & 0.729 & 0.680 \\
   $2019_{RTS{\text -}dynamic}$ & 194   & 111   & 73    & 0.636 & 0.727 & 0.678 \\
%   $2017_{time{\text -}IOU}$ & 192   & 113   & 74    & 0.630 & 0.722 & 0.673 \\
%   $2018_{time{\text -}IOU}$ & 194   & 111   & 72    & 0.636 & 0.729 & 0.680 \\
%   $2019_{time{\text -}IOU}$ & 194   & 111   & 73    & 0.636 & 0.727 & 0.678 \\

\bottomrule
    \end{tabular}
%\vspace{1ex}
%\raggedright \\ \textbf{TP}, \textbf{FP}, and \textbf{FN}: count of true positives, false positive, and false negative, \textbf{IOU}: IOU threshold, \textbf{Pre}: precision, \textbf{Rec}: recall, \textbf{F1}: F1 score
  \label{table_accuracy_f1}
\end{table}


\subsubsection{Improvement when using multi-temporal images}
\label{sec_improve_using_multi_images}

%%%%%%% Mapping results from exp7 %%%%%%%%%%%%%
\begin{figure} 
	\centering
	\includegraphics[width=14cm]{figs/exp7_p_r_curves_trim.jpg}
	\caption{Precision-recall curves and average precision of mapped polygons in 2017, 2018, and 2019. (a) and (b) are precision-recall curves and AP values without and with applying the assumption of RTS dynamic features, respectively.}
	\label{fig_p_r_curve_exp7}
\end{figure}

%Mapping results using multi-temporal remote sensing images, without negative training polygons. 
%Improvement by applying time IOU and occurrence

The F1 scores and AP values of mapped polygons from 2017 to 2019 improve significantly after applying the time-IOU and three-polygon criteria. 
As shown in Table \ref{table_accuracy_f1} and Fig. \ref{fig_p_r_curve_exp7}, F1 scores improve around 0.5, from $\sim$0.2 to $\sim$0.7, and AP values improve from less than 0.1 to around 0.32. 
By utilizing the dynamic features of RTSs in the multi-temporal images, the number of false positives reduces dramatically, from around 2000 to 100, as shown in Table \ref{table_accuracy_f1}.
Therefore, the F1 scores and AP values improve significantly.
However, the number of true positives reduce around 40, resulting in the increase of false negatives. 
The reason is that some of the mapped polygons do not well match RTS boundaries, leading to violation with the assumption of RTS dynamic features. 
 
%compared with the ones including negative training polygons 
The assumption of RTS dynamic features cannot contribute to the improvement of mapping accuracies if the initial mapped results contain a small population of false positives. 
As shown in Table S1, the F1 scores only improve a little after applying the RTS dynamic features, however, the AP values (Fig. S2) reduce 0.01 or 0.04 in different years.   
The reasons include: (1) the number of false positives reduce around 70, but TP also reduces around 40, resulting in an almost unchanged F1 score 
and (2) including the same negative training polygons used in \cite{huang2020using} has depressed the population of false positives. 


%TODO: results from k-fold cross-validation (mention it, then put into supplementary files)

\subsubsection{Comparison with one-year results}
\label{sec_compare_with_201805_results}

Both F1 score and AP values of mapped polygons in this paper are a little worse than the accuracy of one-year results presented in \cite{huang2020using}. 
Regarding the F1 scores (Table \ref{table_accuracy_f1}) after applying RTS dynamic features, the difference between the one in \cite{huang2020using} and in this paper are around 0.17, while the difference of AP values are around 0.2.
Even the results including negative training polygons, the F1 score and AP values are smaller than one-year results at around 0.08 and 0.1, respectively. 
One possible reason is related to the variation of features over time. 
For example, the 2018 and 2019 Planet images (Fig. \ref{fig_multi_rts_image_studyarea} c and d) are brighter than the 2017 image (Fig. \ref{fig_multi_rts_image_studyarea}b), which may confuse the Deeplabv3+ model during training. 

%Compared the precision recall curves, the AP value is smaller, because we don't have the high values of recall and precision. 


\subsection{RTS expansion from 2017 to 2019}
\label{sec_rts_expanding}

% the figure here, shows change area and retreat distance, are derived from manual delineation.
\begin{figure} 
	\centering
	\includegraphics[width=14cm]{figs/rts_change_area_dis_manu_trim.jpg}
	\caption{Histogram of RTS expansion area and retreat distance from 2017 to 2019. (a) and (b) are expanding area  of 2017 vs 2018 and 2018 vs 2019, respectively. An RTS with expanding area of 2.53 ha is not shown in  (b). (c) and (d) are retreat distance from 2017 to 2018 and 2018 to 2019, respectively.}
	\label{fig_rts_change_area_reDis}
\end{figure}

\subsubsection{The area of RTS expansion}
\label{sec_rts_change_area}

% In the manual delineation results, we have 266, 266, and 267 RTSs in 2017, 2018, and 2019 images, respectively. 
% In the RSE 2020 paper, we identify 263 RTSs (2018 May images).

In the Beiluhe region, most RTS are expanding from 2017 to 2019 with average expanding area of 0.21 ha from 2017 to 2018 and 0.23 ha from 2018 to 2019 as shown in Fig. \ref{fig_rts_change_area_reDis}a and b.
From 2017 to 2018, 219 RTSs are expanding, among them, 77 ones with expanding area greater than 0.2 ha; while from 2018 to 2019, 189 RTSs are expanding, and 66 of them with expanding area greater than 0.2 ha.
The maximum expanding areas are 1.47 ha and 2.53 ha for 2017 vs 2018 and 2018 vs 2019, respectively. 
The ground truths outline 266 RTSs in Beiluhe region, therefore, 82\% of them are expanding from 2017 to 2018 and 71\% from 2018 to 2019. 
As shown in Fig. S3, around half of RTSs only have one expanding branches, while remains (121 for 2017 vs 2018 and 90 for 2018 vs 2019)  has two or more expanding branches. 

%How to define an RTS was expanding?
%How many RTS have expanding? 
%not to define, just say how many has expanding area greater than ...?

%TODO: should I discuss the automatic mapping results? Maybe not here, but in discussion session. 

\subsubsection{RTS retreat distance}
\label{sec_rts_retreat_distance}

%retreat distance from 2017 to 2019. 
%TODO: add standard variation for retreat distance. 
The average retreat distance of RTSs from 2017 to 2018 and 2018 to 2019 is 19.5 m and 20.8 m (Fig.\ref{fig_rts_change_area_reDis}b and c), respectively. 
The maximum retreat distance is 55.6 m and 77.2 m for 2017 vs 2018 and 2018 vs 2019, respectively, based on medial circles. 
From 2017 to 2018, 63\% of RTSs retreat more than 15 m, while from 2018 to 2019, 55\% of them have retreat distance greater than 15 m. 
% measure the retreat distance from manual delineation, mentioned here or in the (discussion) supplementary materials?
%TODO: show planet images in different years for this one
In Fig. S5, the retreat distance based on manual drawn lines is 21.1 and 25.2 for 2017 vs 2018 and 2018 vs 2019, respectively, while the maximum retreat distance is 90.8 m and 212.8 m

%Other measurement,  uncertainty, and standard variation
The three automatic measurement of retreat distance is slightly different with ones derived from manually drawn lines, among them, the one based on the diameter of the maximum medial circle is closest the manual measurement. 
As shown in Fig. S6, the mean difference are $-1.6$ m \& $-4.4$ m, 17.8 m \& 14.8 m, and 14.0 m \&10.6 m (2017 vs 2018 \& 2018 vs 2019) for the one based on diameter of the maximum medical circles, the direction of the maximum elevation difference, and across the centroid of old polygons, respectively. 
%Retreat distance estimated based on diameter of the maximum medical circles tends to be underestimated but a few one them 
As mentioned in section \ref{sec_cal_retreat_dis}, the retreat distance calculated from medial circles tend to be underestimated, but the other two automatic measurements are overestimated. 
As shown in the histogram (Fig. S6), for the retreat distance of some RTSs are greater than the one from manually draw lines. 
The reason is that the manually drawn lines many not exactly match the direction of RTS maximum retreat.  

 % standard deviation
\begin{figure} 
	\centering
	\includegraphics[width=14cm]{figs/standard_deviation_trim.jpg}
	\caption{Histogram for standard deviation of three automatic retreat distance measurement. (a) is the standard deviation for 2017 vs 2018, and (b) for the one of 2018 vs 2019, respectively.}
	\label{fig_re_dis_standard_var}
\end{figure}

The retreat distance of most RTSs measured by the three automatic method are consistent as represented by the standard variation (Fig. \ref{fig_re_dis_standard_var}).
From 2017 to 2018, 97\% RTSs have standard variation smaller than 30, while for 2018 to 2019, 98\% smaller than 30, indicating a good agreement. 

%TODO: compare with other retreat distance reported in other region?  in discussion session.  
 

 

\section{Discussion}
\label{sec_discussion}

%\subsection{Advantage and disadvantage of using the RTS dynamic feature}
%\label{sec_diss_rts_dynamic_features}
\subsection{Delineating RTS boundaries on multi-temporal images}
\label{sec_diss_mapping_rts_multi_images}

Accurate RTS boundaries on multi-temporal images are essential for quantifying their development.
Deep-learning-based method can automatically delineate RTS boundaries on images with high accuracies and is critical for investigation of RTSs in large areas. 
Based on RTS boundaries in different years, we can revel their development in unprecedented spatial and temporal domain. 
As demonstrated in this paper, the results from deep-learning-based method can accurately delineate most RTS boundaries in different years. %, and potentially, can applied to larger areas. 


However, delineation of RTS boundaries on multi-temporal images is more challenging than identification of their location or on one-year images. 
In this study, the targets are not only locations and extents of 266 RTSs in Beiluhe, but also their boundaries in different years, that is, $266\times3$ RTS boundaries from 2017 to 2019.
As shown in section \ref{sec_compare_with_201805_results}, the F1 scores derived from multi-temporal images are little worse than one-year results, indicating the challenges of mapping RTSs in multi-temporal images. 
The possible reasons include (1) the inconsistent features of the same RTS in different images may confuse the trained model 
and (2) inaccurate mapped boundaries cannot meet the requirement of RTS dynamic features. 
These reasons are related to the mapping strategy, that is, training a single model and removing false mapped polygons by applying RTS dynamic features.  
Alternatively, we may adopt other strategies such as training multiple models, a model for each year,  which is still manageable when images only span a few years.
Applying the RTS dynamic feature can significantly reduce the number of false positives (section \ref{sec_improve_using_multi_images}), but it also remove some correct mapped polygons with relatively lower accuracy. 
Another solution is that lowering the criteria of RTS dynamic features or adding some negative training polygons for training if they are available.
One limitation of applying the RTS dynamic features is that a new emergent RTS (after 2018 or 2019) would be removed. 


% how to improve, % talking about different strategies. 
%Advantage: do not need negative training polygons, so save time when extend to larger areas.

\subsection{Automatic calculation of RTS retreat distance}
\label{sec_diss_retreat_distance}

Automatic calculation of RTS retreat distance is necessary for investigation of RTS development in large areas.
Currently, measure of RTS retreat distance is conducted in the field or on images using manual measurement, which is limited to local areas. 
In this study, a proposed method based on polygon-based change detection and medical circles fills the gap of the automatic calculation of RTS retreat distance 
and provides a basis tool for quantifying RTS development in large areas. 
Moreover, the input polygons (RTS boundaries) can be either derived from manual delineation or automatic mapping methods, 
providing the ability to utilize both archived and newly produced data.


The retreat distance of most RTSs are equal or close to manual measurement on images, and only a few of them show big discrepancy due to irregular or elongated shapes of expanding areas. 
%Why we need three automatic measurement for the retreat distance?
Therefore, in additional to the automatic measurement of medical circles, we also conduct other two automatic measurements (based on DEM difference and centroid of the corresponding old polygon) as supplementary measurement. 
For a specific RTS, further investigation may need to conduct if the discrepancy (represented by standard variation) is large. 
These irregular or elongated shape of expanding areas may cause by spatial distribution of ground ice and micro-terrain.
The accuracy of RTS boundaries (input polygons) can affect the accuracy of retreat distance, especially when these boundaries are derived from automatic mapping methods. 
Currently, automatic mapping (multi-temporal or one-year) alway has a few mapped polygon are inaccurate, leading to the uncertainties of retreat distance. 
Therefore, in the further, we may consider using pixel-based change detection to obtain the expanding area directly from multi-temporal images. 



%\subsection{Comparison with RTS retreat rates in the Arctic}
%
%No report of RTS retreat in the Tibetan Plateau, but the one in the Arctic.


\section{Conclusions}
\label{sec_conclusion}

We did two things, (1) automatic mapping on multi-temporal images. 
(2)Find the changes automatically. 

 Contribution of this works: (1) provide an automatic method for calculating the retreat distance (no matter the RTS boundaries are from manual delineation or automatic mapping).
  and (2) 

\section{Data and codes}
\label{sec_data_codes}

Planet images can be downloaded via Planet Website( \url{https://www.planet.com/}). 
Codes are available on GitHub (\url{https://github.com/yghlc/ChangeDet_DL}) after the acceptance of this paper.

\section{Acknowledgments}
\label{sec_acknowledgments}

Thanks to ...


%% The Appendices part is started with the command \appendix;
%% appendix sections are then done as normal sections
%% \appendix

%% \section{}
%% \label{}

\section{References}
\label{sec_reference}

%% If you have bibdatabase file and want bibtex to generate the
%% bibitems, please use
%%
\bibliographystyle{elsarticle-harv} 
%\bibliography{polygon_based_rts_changeDet.bib}
% the bib file in "~/codes/Texpad/shared_files"
\bibliography{permafrost_rs_ref.bib}

%% else use the following coding to input the bibitems directly in the
%% TeX file.

%\begin{thebibliography}{00}
%
%%% \bibitem[Author(year)]{label}
%%% Text of bibliographic item
%
%\bibitem[ ()]{}
%\end{thebibliography}


\end{document}

\endinput
%%
%% End of file `elsarticle-template-harv.tex'.
